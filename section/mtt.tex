\section{Dependency Parsing using MTT}

\subsection*{Definition} Construct dependency tree (DT) with syntactic relations, e.g., noun-modifier, determiner, etc. The additional root node should only have degree 1.

\subsection*{Relation to context free grammar}
CRF has no information about syntactic relation, DT has no information about constituency structure.

\subsection*{Types} 
(1) Projective DT: no crossing arcs. Algorithms are generally dynamic programming.
(2) Non-Projective DT: crossing arcs. Algorithms use matrix-tree theorem (MTT).


\subsection*{Edge-Factored Assumption targeted the difficulty}
Assume $\score(t,w)$ is the product of edge scores (including the root edge).

Define $A_{ij}=\exp(\score(i,j,w))$ be the edge score, $\rho_j = \exp(\score(j,w))$ be the root score. Then MTT says $Z=|L|$ for following $L$ which is $O(n^3)$:

$$
L_{i j}= \begin{cases}\rho_{j} & \text { if } i=1 \\ \sum_{i^{\prime}=1, i^{\prime} \neq j}^{n} A_{i^{\prime} j} & \text { if } i=j \\ -A_{i j} & \text { otherwise }\end{cases}
$$

\subsection*{Decoding the best DT}
Equivalent to find the best directed spanning tree starting from root and the degree of root is 1. Greedy algorithms that work in undirected graph do not work.

We apply Chu-Liu-Edmonds Algorithm which is $O(n^3)$:
(1) Find the best \emph{incoming} edge for each vertex. 
(2) Contract cycles to be a single node $c$ and reweight the \emph{incoming} edge to $c$ by adding the valid weights in $c$ if the edge is chosen. 
(3) The graph now has a spanning tree. If the root constraint is not satisfied, reweight each \emph{outcoming} edge from root to $v$ by subtracting the weight of next best \emph{incoming} edge to $v$. Remove the lowest \emph{outcoming} edge from root and repeat step 3. 
(4) Expand contract nodes by breaking cycles accordingly.